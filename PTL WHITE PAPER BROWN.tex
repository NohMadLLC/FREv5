\documentclass[12pt]{article}
\usepackage{amsmath,amssymb}
\usepackage{geometry}
\usepackage{hyperref}
\usepackage{graphicx}
\geometry{margin=1in}
\hypersetup{
    colorlinks=true,
    linkcolor=blue,
    urlcolor=blue,
    pdftitle={PTL: Principle of Temporal Lensing},
    pdfauthor={Breezon Brown}
}
\title{\bfseries PTL: Principle of Temporal Lensing \\ \large A Universal Model of Time Distortion via Memory, Emotion, and Recursion}
\author{Breezon Brown}
\date{July 2025}

\begin{document}

\maketitle

\begin{abstract}
The Principle of Temporal Lensing (PTL) is a universal, testable model for how subjective time is distorted by memory density, emotional charge, and recursive stability. PTL provides a mathematically robust foundation for quantifying trauma, collapse, and cognitive resilience across both biological and artificial systems.
\end{abstract}

\section*{I. Introduction}
Subjective time distortion has long been observed in trauma, anxiety, and altered states of consciousness, but until now, lacked a quantitative model. The PTL framework is built on three variables: memory density ($M$), emotional charge ($E$), and recursive stability ($R$). The core insight is that time distortion is governed not by an external clock, but by the internal interplay of these factors. 

\section*{II. Core PTL Equation}
The fundamental equation is:
\[
T' = \alpha \cdot \exp\left(-\beta \cdot \frac{C \cdot M^{\kappa} \cdot E^{\epsilon}}{R^{\gamma}}\right)
\]
where:
\begin{itemize}
    \item $T'$ = Subjective time distortion (dimensionless)
    \item $M$ = Memory density (normalized, $0\leq M\leq 1$)
    \item $E$ = Emotional charge (normalized, $0\leq E\leq 1$)
    \item $R$ = Recursive stability (entropy-based, $R>0$)
    \item $C$ = Cultural coherence (dimensionless, optional; $C=1$ if omitted)
    \item $\alpha$ = Scaling constant
    \item $\beta$ = Universal time distortion constant
    \item $\kappa, \epsilon, \gamma$ = power-law exponents for $M, E, R$
\end{itemize}

\section*{III. Variable Definitions}
\begin{itemize}
    \item \textbf{Memory Density ($M$):} Quantifies the amount and accessibility of encoded episodic or procedural memory.
    \item \textbf{Emotional Charge ($E$):} Quantifies the mean amplitude of affective response during the period measured.
    \item \textbf{Recursive Stability ($R$):} Defined by the normalized entropy of recursive/feedback neural dynamics:
    \[
    R = \left[-\int_0^T \Phi(t) \ln(\Phi(t) + 10^{-12})\,dt\right]^\gamma
    \]
    where $\Phi(t)$ is normalized neural activity at time $t$.
    \item \textbf{Cultural Coherence ($C$):} (Optional) Quantifies the degree to which collective memory and meaning structure are shared by a social group.
\end{itemize}

\section*{IV. Empirical Roadmap}
\begin{enumerate}
    \item Measure $M$ and $E$ via standardized self-report and psychometric tools (e.g. memory recall tests, affective surveys).
    \item Record $\Phi(t)$ using EEG, fMRI, or time-series neural data.
    \item Calculate $R$ using the entropy integral above.
    \item Fit $\alpha, \beta, \kappa, \epsilon, \gamma$ to observed subjective time distortion $T'$ in controlled experiments.
\end{enumerate}

\section*{V. Collapse Threshold and PTSD}
PTL predicts a trauma-collapse regime when $R$ falls below a critical threshold and $E$ spikes. This matches clinical PTSD time distortion. For diagnostic purposes, set $\gamma=1.0$, $C=1.0$, and estimate $\beta$ empirically from cohort data.

\section*{VI. Model Validation and Use Cases}
\begin{itemize}
    \item \textbf{Clinical:} PTSD, anxiety, dissociation diagnostics; therapy monitoring.
    \item \textbf{Artificial:} AI time perception, hallucination detection, recursive agent modeling.
    \item \textbf{Cultural:} Generational trauma, historical time lensing in collective memory.
\end{itemize}

\section*{VII. Copyright, Ethics, and Licensing}
\vspace{-1em}
\hrule
\smallskip
\textcopyright\, 2025 Breezon Brown (NohMad LLC).  
No part of this framework may be reproduced or used for commercial or derivative AI systems without written permission.  
All research, code, and derivations remain the exclusive property of Breezon Brown and NohMad LLC.  
Use of PTL or any derivative system for medical, AI, or commercial purposes requires explicit licensing.
\hrule

\end{document}
